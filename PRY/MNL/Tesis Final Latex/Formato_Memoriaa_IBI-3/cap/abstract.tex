The translation of the genetic information contained in a messenger RNA (mRNA) is an essential process for the cell, which is highly regulated, both in eukaryotes and prokaryoic cells. In bacteria, the rate of translation of mRNA depends on the availability of the repertoire of aminoacyl - tRNAs that recognize each of their codons, the mRNA structure and interaction with other molecules such as proteins or RNAs. Small RNAs (sRNA or small RNA) are a group of small (50 to 400 nucleotides), noncoding RNAs, which modify the rate of translation of mRNA through interaction with the coding region or with the UTR ends. The reasons why some genes are regulated by sRNAs are not clear, however, that some interactions sRNA-mRNA occur in coding regions, makes it feasible to propose that regulation by sRNA and the properties which depends of codon composition such the adaptation to the tRNA pool are closely related. This thesis proposes to study in \textit{E. coli} K12 the sRNA-mRNA interactions present to identify whether genes regulated by sRNAs present in their coding region or in the region of interaction with the sRNA a set of codons with particular tRNA availability. Furthermore, we propose to study how this control is integrated with the metabolism of \textit{E. coli} forming a regulatory network. To develop this thesis we suggest as hypothesis: "In \textit{Escherichia coli} K12 there is a network of regulation mediated sRNA that integrates the various metabolic processes, in which the regulated sRNA genes exhibit a pattern of codons and adaptation pool of different tRNA to present in genes not regulated by sRNAs". To test this hypothesis we propose specific objectives 1. To identify by searching databases couples of sRNAs and mRNA in \textit{Escherichia coli} K-12 MG1655. 2.- To identify and to predict the interaction regions between mRNA and sRNA. 3.- To characterize the adaptation to the tRNA repertoire of the genes regulated by sRNA and its interaction regions. 4.- To generate a map of sRNA-mRNA interactions that includes the metabolic network to which the mRNA belongs, and the adaptation to codogenic use and repertoire of tRNA, both of the complete gene and of the interaction region.

We identified in \textit{Escherichia coli} K12 several sRNA-mRNA networks using interactions determined by empirically and predictive methods. This networks also includes the efficiency with which the genes are translated. Our results suggest that in Escherichia coli K12, the regulatory mechanism by small RNA, occurs in regions with codons highly represented by low availability of tRNA. This implies that the regulation by sRNA occurs in regions where the rate and advancement of the ribosome is lower.
