\chapter{Discusi\'{o}n}

\section{Discusi\'on}

La traducci\'on de los mensajeros a prote\'inas es un complejo proceso biol\'ogico en el cual los sRNA act\'uan como reguladores afectando este proceso de manera positiva o negativa dependiendo el tipo de interacci\'on que conforme. La traduccci\'on tambien es afectada por la composisci\'on codog\'enica de la regi\'on codificante. Genes que presentan codones que poseen una mayor disponibilidad de tRNA son traducidos mas eficientementes que aquellos compuestos genes que poseen codones de baja frecuencia o con baja disponibilidad de tRNAs que los decodifican. Para predecir la eficiencia de la traducci\'on de un gen existen 2 \'indices el CAI y tAI, siendo el CAI el que evalua la adaptaci\'on codog\'enica y tAI evalua los tRNAs disponibles para cada cod\'on. Con la finalidad de establecer una relaci\'on entre el tAI y CAI de un gen con su regulaci\'on por sRNA se estudiaron los CAI y tAI de los mensajeros que interaccionan con sRNA y se compararon con los valores de CAI y tAI de los mensajeros que no interactuaban con sRNAs con la finalidad de establecer si la interacci\'on sRNA-mRNA esta directamente relacionada con los valores de estos \'icomo ndices.

Para demostrar esto se plante\'o la siguiente hip\'otesis: En Escherichia coli K12 existe una red de regulaci\'on mediada por sRNA, en la cual los genes regulados por sRNA presentan un patr\'on de codones y adaptaci\'on al pool de tRNA diferente al presente en los genes no regulados por sRNAs. Con la finalidad de responder a esta hip\'otesis se plantearon los siguientes objetivos:

\begin{enumerate}
	\item Identificaci\'on mediante b\'usqueda en las bases de datos las parejas de sRNAs y mRNA en \textit{Escherichia coli} K-12 MG1655.
	
	\item Identificaci\'on y predicci\'on de las regiones de interacci\'on entre mRNA y sRNA.
	
	\item Caracterizaci\'on de la adaptaci\'on al repertorio de tRNA de los genes regulados por sRNA y de sus regiones de interacci\'on.
	
	\item Generaci\'on de un mapa de interacciones sRNA-mRNA que incluya la adaptaci\'on del repertorio de tRNA y uso codog\'enico del gen completo y de la regi\'on de interacci\'on.
\end{enumerate}

Respecto al OE1 a partir de los datos obtenidos en las bases de datos se identificaron 108 sRNAs en \textit{E. coli}, de los cuales 96 presentaron interacciones sRNA-mRNA con 304 mensajeros y 12 no registraron ning\'un tipo de interacci\'on. Cabe destacar que en este punto se identific\'o una gran cantidad de interacciones sRNA-mRNA presentes bibliograf\'ia en las cuales se conoce el sRNA y su mRNA blanco, pero no la zona donde ocurre esta interacci\'on. Para lograr esto se utilizaron 2 bases de datos sRNATarBase y BSRD.
sRNATarBase es una base de datos que contiene la informaci\'on de interacciones sRNA-mRNA de 17 organismos bacterianos \cite{Cao2010}, donde se encuentra la informaci\'on de \textit{E. coli} K-12 MG1655, de las cuales se encuentran 230 interacciones de las cuales algunas de estas se les conoce el sRNA y su blanco, pero no la zona en donde ocurre esta interacci\'on. Por otra parte BSRD es una base de datos que contiene la informaci\'on de 783 especies bacterianas que dan cuenta de 957 interacciones\cite{Li}. Para \textit{E. coli} K-12 MG1655, en esta base de datos est\'an contenidos todos los sRNAs identificados para este organismo, junto con esto tambi\'en se encuentran las interacciones extra\'idas de sRNATarBase, e interacciones predichas con la identificaci\'on del software que se utiliz\'o, como tambi\'en se incluye la energ\'ia de la posible interacci\'on. Por otra parte, la base de datos creada en este trabajo (DB\_Ecoli\_V3) es una base de datos que solo se encarga de almacenar la informaci\'on de las interacciones sRNA-mRNA de \textit{E. coli} K-12 MG1655, contiene informaci\'on relevante como la energ\'ia de interacci\'on, el m\'etodo que se utiliz\'o para describir la interacci\'on, el efecto, entre otros. En comparaci\'on con sRNATarBase y BSRD, DB\_Ecoli\_V3 es m\'as completa en relaci\'on a la informaci\'on de las interacciones de \textit{E. coli} K-12 MG1655 ya que esta base contiene todos los datos relevantes de los sRNAs (nombre sin\'onimo y divisi\'on de zonas del sRNA), como tambi\'en de los mensajeros (secuencia del mensajero con las zonas UTR, Valores de CAI y TAI).

En el caso del OE2 a partir de los resultados obtenidos anteriormente se procedi\'o a caracterizar las interacciones que faltaban y a predecir las interacciones para los sRNAs que faltaban, esto llev\'o a poder completar de esta manera a todos los sRNAs que est\'an presentes \textit{E. coli}, por lo tanto, qued\'o un total de 435 interacciones con un total de 326 mensajeros, divididos en 322 interacciones predichas (equivalente a un 59,31\%) y a 113 interacciones emp\'iricas (equivalentes a un 40,68\%). Cabe mencionar que la identificaci\'on de los mensajeros con sRNAs llev\'o al proceso de identificaci\'on de zona de interacci\'on en los sRNAs, como lo estableci\'o Taylor B. Updegrove y otros en su art\'iculo \cite{Updegrove2015}, que a diferencia de su investigaci\'on se utilizaron todos los mensajeros que interaccionan con un sRNA en espec\'ifico, dando de esta manera para el caso de Spot42 un resultado muy similar al que ellos obtuvieron por medio de su investigaci\'on.

La metodolog\'ia creada por Taylor B. en su art\'iculo\cite{Updegrove2015}, la cual utiliza la energ\'ia libre de Gibbs por nucle\'otido para obtener un patr\'on de interacci\'on en el sRNA. Esta metodolog\'ia se utiliz\'o en investigaciones como la que estudia al sRNA GlmZ el cual es dependiente de Hfq y modula la traducci\'on de glmS [38] y en otro estudio para comprobar la dependencia de la prote\'ina ProQ para el sRNA RaiZ en Salmonella Typhimurium\cite{Smirnov2017}. En comparaci\'on la metodolog\'ia desarrolla por \'Avalos F, Tello M y otros, utiliza un script el cual comprende el largo del sRNA y marca con 1 y con 0 en donde hay interacci\'on sRNA-mRNA, la ventaja de este m\'etodo es que utiliza todos los mensajeros que interact\'uan con el sRNA, siendo el \'unico par\'ametro para crear el patr\'on de interacci\'on la posici\'on de la interacci\'on en el sRNA, esto despu\'es se gr\'afica y se procede a identificar las zonas de interacci\'on en el sRNA. La ventaja de esta metodolog\'ia es la creaci\'on de la red de interacci\'on a partir del gr\'afico en la cual se puede observar como el mensajero interact\'ua con el sRNA. La desventaja de esta metodolog\'ia en comparaci\'on con la desarrollada por Updegrove es la cantidad de pasos que fueron necesarios para llevar a un resultado similar.

Para el caso del OE3 se realiz\'o an\'alisis estad\'isticos con la finalidad de encontrar diferencias entre los valores de CAI y tAI para los genes regulados y no regulados por sRNAs, lo cual nos mostr\'o para el caso de este tipo de comparaci\'on ninguna diferencia significativa, sin embargo, al comprar las sitios de interacci\'on que son regulados por sRNAs se logran encontrar diferencias en esto sobre todo en las regiones 5'CDS y 3'CDS para el caso del valor de CAI, es decir que estas regiones tienen la tendencia de utilizar codones que se encuentran con mayor frecuencia a lo largo del genoma, para el caso del valor de tAI se encontr\'o que la regi\'on de interacci\'on presenta valores de este \'indice menores, lo que sugiere que los codones utilizados para formar interacci\'on son de un alto uso pero que presenta un poca cantidad de tRNA para realizar la traducci\'on.

En el proceso de traducci\'on los ribosomas son los encargados de trasformar los codones de la secuencia codificante en amino\'acidos. Durante este proceso la velocidad de traducci\'on es variable dependiendo del cod\'on que est\'e presente en la secuencia codificante, es decir, si el cod\'on que se está traduciendo tiene un alto uso y a la vez una alta disponibilidad de tRNA la velocidad de traducci\'on es alta, en el caso de que el cod\'on sea de alto uso y a la vez tenga una baja disponibilidad de tRNA la velocidad de traducci\'on ser\'a m\'as lenta. La forma de evaluar esto es por medio de los \'indices de CAI y tAI y de esta manera se podr\'ia identificar las posibles zonas del \'area codificante en donde la traducci\'on avanza m\'as lento. La identificaci\'on de zonas de baja velocidad de traducci\'on podr\'ia ayudar a encontrar nuevos sRNAs que interacciones con esta, ya que los sRNAs modulan la traducci\'on en zonas del \'area codificante que presente codones de alto uso y baja disponibilidad de tRNAs ya que al presentar el ribosoma una menor velocidad de traducci\'on le entrega una mayor cantidad de tiempo al sRNA para formar la interacci\'on sRNA-mRNA y de esta manera lograr reprimir el proceso de traducci\'on de este. Este principio puede ser utilizado para identificar nuevas interacciones sRNA-mRNA en otros organismos bacterianos como por ejemplo bacterias que afectan a especies de pescados (como, por ejemplo: \textit{Salmo Salar, Oncorhynchuis mykiss}, entre otras especies acu\'aticas), como tambi\'en en otras cepas de \textit{E. coli}, como tambi\'en en distintas cepas de \textit{Salmonella} o cualquier organismo bacterianas que presenten en su genoma RNA no codificante (espec\'ificamente sRNAs).

En el caso del OE4 para expresar los resultados obtenidos se cre\'o una red de interacci\'on en donde se pudieron identificar todas las interacciones registradas para \textit{E. coli}, junto con esto se crearon redes de interacci\'on para cada sRNA de manera individual, lo que ayud\'o a identificar la organizaci\'on de las interacciones por medio del CAI y tAI que presentan los mensajeros que interaccionan con estos sRNAs en particular y como se distribuye el CAI y el tAI de los mensajeros a lo largo del sRNA y como existen zonas los sRNAs que tienen preferencias espec\'ificas de interacci\'on con mensajeros que presentan ciertos valores de tAI y CAI. Este proceso se podr\'ia expandir hacia otros organismos bacterianos con la finalidad de comprender como se comportan los valores de CAI y tAI de los mensajeros que interaccionan con sRNAs de estos organismos en espec\'ifico.

Para el estudio de interacciones sRNA-mRNA la creaci\'on de redes de interacci\'on es importante, ya que por este m\'etodo de representaci\'on gr\'afica se puede tener una visi\'on general la forma en que los sRNAs interact\'uan con mensajeros. Varios autores han logrado representar las interacciones sRNA-mRNA de manera exitosa por medio del uso de redes de interacci\'on sRNA-mRNA, tomemos por ejemplo el caso de los autores Stortz et. al \cite{Storz2011} en el cual representaron las interacciones de 20 sRNAs (lo cual equivale a 18.5\% de los sRNAs identificados) con 99 mensajeros, en el cual solo muestra la sRNA con su blanco, no muestran informaci\'on como el efecto de dicha interacci\'on, si la interacci\'on es emp\'irica o predicha, etc. En el caso de la red de interacci\'on elaborada por \'Avalos, F., Tello, M, la red de interacci\'on elaborada por ellos muestra a los 108 sRNAs con sus respectivos blancos, el efecto de la interacci\'on, si la interacci\'on que se describe es emp\'irica o predicha, junto con esto tambi\'en incluye los valores de CAI y tAI representados por color y tama\~no del nodo del mensajero respectivamente. En este aspecto la red de interacci\'on descrita por los dos autores anteriores muestra mucha m\'as informaci\'on que la primera red que describ\'ia la interacci\'on sRNA-mRNA en \textit{E. coli}, abriendo una nueva ventana hacia la comprensi\'on de las interacciones sRNA-mRNA.